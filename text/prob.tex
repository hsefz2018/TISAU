\documentclass[UTF8, 11pt, a4paper]{article}
\usepackage[cm]{sfmath}
\usepackage{tabularx}
\def\arraystretch{1.3}
\usepackage[a4paper, top=3.18cm,bottom=3.81cm,left=2.54cm,right=2.54cm]{geometry}
\usepackage{indentfirst}
\setlength{\parskip}{6pt}
\XeTeXlinebreaklocale "zh"
\usepackage{graphicx}
\usepackage[normalem]{ulem}

\usepackage{fontspec}
\setmainfont{思源黑体}
\SetSymbolFont{largesymbols}{normal}{OMX}{iwona}{m}{n}
\setmonofont{Source Code Pro}

\begin{document}

\title{HSEFZ Practice Contest}
\maketitle

* 时间:3 小时
\newpage

\section*{Technology Intelligence See All Unbounded}

世界并非如你所见。

每一个存在着文明的角落,都散布着一种神秘的、未曾知晓的能量——研究者们称之为“异域能量”%
(Exotic Matter,缩写为 XM)。而人类的文明,也或多或少地在 XM 的影响下,在历史的长河中%
书写着属于它的篇章。

欧洲核子研究委员会(CERN)的科学家们聚到一起,成立了一个名为 Niantic Project 的项目,%
专门进行 XM 相关的研究。种种证据表明,XM 的背后存在着某种不可名状的强大力量……%
但关于它到底会给人类文明带来什么,没有人敢妄下论断。

科学家们最终找到了一种将智能手机转变成为“XM 扫描器”的方法……当这个实验室中的程序被%
匿名者伪装、改名为“Ingress”并公之于众,大量用户和 XM 发生了接触,成为了这场实验的参与者。%
人类内部反对和支持公开 XM 的分歧日益加剧,最终形成抵抗者(Resistance)和启蒙者(Enlightened)%
两个阵营,两方观点的拥护者成为了为各自阵营奋斗的特工……

XM 究竟会将人类引向何方?毁灭、启蒙,抑或是重生?

无论选择接受还是抵抗,都是时候行动了。
\newline\newline

\subsection*{题目概况}
\begin{tabularx}{\textwidth}{|X|X|X|X|}
\hline
题目名称 & A & B & C \\ \hline
文件名 & anomalies.* & portals.* & links.* \\ \hline
每个测试点时限 & 1 秒 & 1.5 秒 & 1 秒 \\ \hline
测试点数目 & 20 & 20 & 50 \\ \hline
每个测试点分值 & 5 & 5 & 2 \\ \hline
运行内存上限 & 256 MB & 256 MB & 256 MB \\ \hline
结果比较方式 & \multicolumn{3}{c|}{全文比较(过滤行末空格及文末回车)} \\ \hline
\end{tabularx}

* 除特殊说明外,输入、输出文件的数字之间均以恰一个空格分隔,行末字符为
Line Feed(\texttt{'\textbackslash n'},ASCII 10),且该字符也存在于输入文件的最后一行末尾。

\subsection*{编译命令(以第一题为例)}
\begin{tabularx}{\textwidth}{|X|c|c|c|}
\hline
C & \multicolumn{3}{l|}{\texttt{gcc -o anomalies anomalies.c -lm}\makebox[5em]{}} \\ \hline
C++ & \multicolumn{3}{l|}{\texttt{g++ -o anomalies anomalies.cpp -lm}\makebox[5em]{}} \\ \hline
Pascal & \multicolumn{3}{l|}{\texttt{fpc anomalies.pas}\makebox[5em]{}} \\ \hline
\end{tabularx}
\newpage

% ========== Problem A ==========

\section*{A. \makebox[1em]{} Anomalies \makebox[2.5em]{} \small{「anomalies.c/cpp/pas」}}
在世界平静的外表下,XM 依旧以它不可知晓的规律运动着。%
全球各地的大城市由于人口密集,往往能吸引大量 XM,而如此数量的 XM 不时会突然出现%
无规律的猛烈扰动——人们称之为“XM 异常”。

令研究者们费解的是,每次 XM 异常都恰好发生在互不相同的三个城市。%
异常的发生时常伴随着一些东西吸引着每一位心怀大志的特工——丢失的数据、黑曜石之盾的碎片……%
因此,两个阵营总会在异常到来之时展开激烈的争夺。

Pisces 居住在一个名为新加坡的城市。根据 Niantic Project 的预测,%
新加坡即将迎来发生在这里的又一次异常,伴随着恢复人工智能 ADA 的数据和能量。%
为了在异常中作出更多的贡献,Pisces 收集了过去所有异常发生的地点并进行了分析。

每一座城市都有一个独一无二的编号,其中新加坡的编号为 $1$。当一次异常在三个城市发生,%
证明这三个城市存在着或多或少的未知联系(当然 Pisces 只需要关注其他城市与新加坡的联系)。%
Pisces 用“相关度”值来衡量联系的强度。这个值由以下规则确定:
\begin{itemize}
    \item 新加坡的相关度为 $0$;
    \item 如果一座城市和另一座相关度为 $x$ 的城市同时发生过一次异常,%
        那么这座城市的相关度不会大于 $x + 1$;
    \item 一座城市的相关度是所有可能取值中的最大值(这个值可以为无穷大)。
\end{itemize}

\subsection*{任务}
给定 $N$ 次异常的发生地点,计算每座城市和新加坡的相关度值。

\subsection*{输入 \makebox[0.5em]{} \small{「anomalies.in」}}
\begin{itemize}
    \item 第 $1$ 行:一个正整数,XM 异常的发生次数 $N$。
    \item 第 $2 \sim N+1$ 行:第 $i+1$ 行包含三个互不相同的正整数,%
        发生第 $i$ 次异常的三个城市的编号 $G_{i,1}, G_{i,2}, G_{i,3}$。
\end{itemize}

\subsection*{输出 \makebox[0.5em]{} \small{「anomalies.out」}}
\begin{itemize}
    \item 第 $1$ 行起:%
        在所有发生过异常的城市中,按\uwave{编号从小到大}的顺序,%
        每座城市占一行,包含两个正整数,依次为城市的编号和城市的相关度。%
        如果相关度为无穷大,输出 $-1$。
\end{itemize}
\newpage

\subsection*{样例}
\subsubsection*{样例一}
\begin{table}[h]\centering
\begin{tabularx}{0.8 \textwidth}{|X|X|}
\hline
\texttt{\textbf{anomalies1.in}} & \texttt{\textbf{anomalies1.out}} \\ \hline
{\ttfamily
7\newline
1 2 3\newline
4 2 5\newline
4 6 5\newline
7 1 2\newline
8 7 9\newline
10 8 11\newline
12 13 14
} & {\ttfamily
1 0\newline
2 1\newline
3 1\newline
4 2\newline
5 2\newline
6 3\newline
7 1\newline
8 2\newline
9 2\newline
10 3\newline
11 3\newline
12 -1\newline
13 -1\newline
14 -1
}
\\ \hline
\end{tabularx}\end{table}

\subsubsection*{样例二}
见附件中给出的 \texttt{\textbf{anomalies2.in}} 和 \texttt{\textbf{anomalies2.out}}。

\subsection*{数据规模与约定}
对于所有测试点,有 $N \geq 1, G_{1, 1} = 1$。

\begin{table}[h]\centering
\begin{tabularx}{0.85 \textwidth}{X|X|X|X} \hline
测试点编号 & $N$             & $\max{G_{i,j}}$ & 附加条件 \\ \hline\hline
1 - 3      & $\leq 10$       & $\leq 50$       & - \\ \hline
4 - 8      & $\leq 1\,000$   & $\leq 3\,000$   & - \\ \hline
9 - 14     & $\leq 150\,000$ & $\leq 450\,000$ & - \\ \hline
15         & $= 150\,000$    & $= 450\,000$    & $G_{i,j}$ 互不相同 \\ \hline
16 - 20    & $\leq 150\,000$ & $\leq 10^9$     & - \\ \hline
\end{tabularx}
\end{table}
\newpage

% ========== Problem B ==========

\section*{B. \makebox[1em]{} Portals \makebox[2.5em]{} \small{「portals.c/cpp/pas」}}
地球上有不少 XM 大量产出的地方,这些地方大都有名胜古迹、雕塑、喷泉或邮局——它们被称为“门泉”。

经过多年的努力,Libra 终于悟出了产生门泉的真谛——放置雕塑!%
Libra 计划在街区内放置一些雕塑并使之成为门泉,来让他和友军收集到更多的 XM。

街区被分成了 $N$ 行 $M$ 列的方格,每个格子内可以是以下三种之一:
\begin{itemize}
    \item \texttt{"\textbf{*}"}:空地,可以放置雕塑;
    \item \texttt{"\textbf{X}"}:障碍物,不可以放置雕塑;
    \item \texttt{"\textbf{O}"}:一位友军的住所,不可以放置雕塑;
\end{itemize}

每个雕塑都可以成为一个门泉,门泉能在地图上曼哈顿距离不超过 $R$ 的所有格子内产生%
一个单位的 XM。每位友军可以收集其住所所在格子的全部 XM。然而毕竟雕塑是有限的,%
Libra 希望在放置不超过 $T$ 个雕塑的前提下,使友军收集到尽可能多的 XM。

第 $r_1$ 行、第 $c_1$ 列的格子与 $r_2$ 行、第 $c_2$ 列格子的曼哈顿距离为%
$|r_1 - c_1| + |r_2 - c_2|$。

\subsection*{任务}
给定一个街区的地图,计算 Libra 用不超过 $T$ 个雕塑最多能使友军收集到多少单位的 XM。

\subsection*{输入 \makebox[0.5em]{} \small{「portals.in」}}
输入的第一行包含一个整数 $C^*$ 表示该测试点包含的数据组数。%
不同组数据互相独立,两组数据间不包含多余的空行。

每组数据的输入如下:
\begin{itemize}
    \item 第 $1$ 行:四个正整数,依次为:街区的行列数 $N$、$M$;
        Libra 拥有的雕塑数量 $T$;门泉产生 XM 的范围 $R$。
    \item 第 $2 \sim N+1$ 行:每行 $M$ 个字符,描述整个街区 $N$ 行 $M$ 列的地图。
\end{itemize}

\subsection*{输出 \makebox[0.5em]{} \small{「portals.out」}}
\begin{itemize}
    \item 第 $1 \sim C^*$ 行:%
        第 $i$ 行包含一个正整数,表示第 $i$ 组数据中友军能收集到 XM 的最多单位数。
\end{itemize}
\newpage

\subsection*{样例}
\subsubsection*{样例一}
\begin{table}[h]\centering
\begin{tabularx}{0.8 \textwidth}{|X|X|}
\hline
\texttt{\textbf{portals1.in}} & \texttt{\textbf{portals1.out}} \\ \hline
{\ttfamily
2\newline
3 3 2 1\newline
*O*\newline
OXO\newline
***\newline
4 5 3 2\newline
**O**\newline
*OXO*\newline
*****\newline
**O**
} & {\ttfamily
4\newline
7
}
\\ \hline
\end{tabularx}\end{table}

在第一组数据中,将两个雕塑放置于第 1 行第 1、3 列的格子可以为友军提供 4 单位的 XM。

在第二组数据中,将三个雕塑放置于第 3 行第 3 列和第 1 行第 2、4 列可以为友军提供 %
7 单位的 XM。也存在另外的方案可以提供同样数目的 XM。

\subsubsection*{样例二}
见附件中给出的 \texttt{\textbf{portals2.in}} 和 \texttt{\textbf{portals2.out}}。

\subsection*{数据规模与约定}
对于所有测试点,有 $C^* = 5$;对于所有数据有 $1 \leq N, M$,$1 \leq T, R \leq 10^8$,%
且至少存在一个空地格子。

\begin{table}[h]\centering
\begin{tabularx}{0.85 \textwidth}{X|X|X} \hline
测试点编号 & $N, M$     & 附加条件 \\ \hline\hline
1 - 3      & $= 4$      & $T = 16$ \\ \hline
4 - 6      & $= 5$      & $T = 5$ \\ \hline
7 - 10     & $\leq 50$  & - \\ \hline
11 - 14    & $\leq 500$ & $N = 1$ \\ \hline
15 - 16    & $\leq 500$ & $R = 1$ \\ \hline
17 - 20    & $\leq 500$ & - \\ \hline
\end{tabularx}
\end{table}

\end{document}

