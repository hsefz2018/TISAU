\documentclass[UTF8, 11pt, a4paper]{article}
\usepackage[cm]{sfmath}
\usepackage{tabularx}
\def\arraystretch{1.3}
\usepackage[a4paper, top=3.18cm,bottom=3.81cm,left=2.54cm,right=2.54cm]{geometry}
\usepackage{indentfirst}
\setlength{\parskip}{6pt}
\XeTeXlinebreaklocale "zh"
\usepackage{graphicx}
\usepackage[normalem]{ulem}

\usepackage{fontspec}
\setmainfont{思源黑体}
\SetSymbolFont{largesymbols}{normal}{OMX}{iwona}{m}{n}
\setmonofont{Source Code Pro}

\begin{document}

\title{HSEFZ Practice Contest}
\maketitle

* 时间:3 小时
\newpage

\section*{Epiphany @ EFZ}

世界并非如你所见。

每一个存在着文明的角落,都散布着一种神秘的、未曾知晓的能量——研究者们称之为“异域能量”%
(Exotic Matter,缩写为 XM)。而人类的文明,也或多或少地在 XM 的影响下,在历史的长河中%
书写着属于它的篇章。

欧洲核子研究委员会(CERN)的科学家们聚到一起,成立了一个名为 Niantic Project 的项目,%
专门进行 XM 相关的研究。种种证据表明,XM 的背后存在着某种不可名状的强大力量……%
但关于它到底会给人类文明带来什么,没有人敢妄下论断。

科学家们最终找到了一种将智能手机转变成为“XM 扫描器”的方法……当这个实验室中的程序被%
匿名者伪装、改名为“Ingress”并公之于众,大量用户和 XM 发生了接触,成为了这场实验的参与者。%
人类内部反对和支持公开 XM 的分歧日益加剧,最终形成抵抗者(Resistance)和启蒙者(Enlightened)%
两个阵营,两方观点的拥护者成为了为各自阵营奋斗的特工……

XM 究竟会将人类引向何方?毁灭、启蒙,抑或是重生?

无论选择接受还是抵抗,都是时候行动了。
\newline\newline

\subsection*{题目概况}
\begin{tabularx}{\textwidth}{|X|X|X|X|}
\hline
题目名称 & A & B & C \\ \hline
文件名 & anomalies.* & portals.* & links.* \\ \hline
每个测试点时限 & 1 秒 & 1.5 秒 & 1 秒 \\ \hline
测试点数目 & 20 & 20 & 50 \\ \hline
每个测试点分值 & 5 & 5 & 2 \\ \hline
运行内存上限 & 256 MB & 256 MB & 256 MB \\ \hline
结果比较方式 & \multicolumn{3}{c|}{全文比较(过滤行末空格及文末回车)} \\ \hline
\end{tabularx}

\subsection*{编译命令(以第一题为例)}
\begin{tabularx}{\textwidth}{|X|c|c|c|}
\hline
C & \multicolumn{3}{l|}{\texttt{gcc -o anomalies anomalies.c -lm}\makebox[5em]{}} \\ \hline
C++ & \multicolumn{3}{l|}{\texttt{g++ -o anomalies anomalies.cpp -lm}\makebox[5em]{}} \\ \hline
Pascal & \multicolumn{3}{l|}{\texttt{fpc anomalies.pas}\makebox[5em]{}} \\ \hline
\end{tabularx}
\newpage

% ========== Problem A ==========

\section*{A. Anomalies \makebox[2.5em]{} \small{「anomalies.c/cpp/pas」}}
Problem Description

\subsection*{任务}
Task Description

\subsection*{输入 \makebox[0.5em]{} \small{「anomalies.in」}}
\begin{itemize}
    \item 第 $1$ 行:Nothing
\end{itemize}

\subsection*{输出 \makebox[0.5em]{} \small{「anomalies.out」}}
\begin{itemize}
    \item 第 $1$ 行:Nothing
\end{itemize}

\subsection*{样例}
\subsubsection*{样例一}
\begin{table}[h]\centering
\begin{tabularx}{0.8 \textwidth}{|X|X|}
\hline
\texttt{\textbf{anomalies1.in}} & \texttt{\textbf{anomalies1.out}} \\ \hline
{\ttfamily
1
} & {\ttfamily
1
}
\\ \hline
\end{tabularx}\end{table}

\subsection*{数据规模与约定}
对于所有任务,有 $T \geq 1, G_{1, 1} = 1$。

\begin{table}[h]\centering
\begin{tabularx}{0.85 \textwidth}{X|X|X|X} \hline
测试点编号 & $T$         & $\max{G_{i, j}}$ & 附加条件 \\ \hline\hline
24         & $= 50\,000$ & $\leq 150\,000$  & - \\ \hline
25         & $= 50\,000$ & $\leq 10^9$      & - \\ \hline
\end{tabularx}
\end{table}

\end{document}

